% Chapter 4

\chapter{Justificación} % Write in your own chapter title
\label{Chapter4}
\lhead{Capítulo 4. \emph{Justificación}} % Write in your own chapter title to set the page header
%Por que resolver el problema es bueno?%
\section{Aceleración del Procesamiento}

Una plataforma de procesamiento concurrente como una FPGA provee aceleración del procesamiento según la configuración adoptada por el desarrollador. Dependiendo de la cantidad de hardware asociado y el diseño del algoritmo, una FPGA es capaz de acelerar varias veces los resultados por unidad de tiempo a diferencia de su contraparte software, como ocurre por ejemplo, en casos como la ejecución de consultas a bases de datos, criptografía, simulaciones, entre otros \citep{salami2017axledb,liu2017elliptic,mingas2016particle,FogEngine}.

\section{Reducción de Costos}

Llevar parte de los servicios de la nube al campo implica la reducción de costos operativos. Esta migración, conocida como computación de borde, implica para el desarrollador invertir más recursos en servidores locales de procesamiento y almacenamiento de datos que tienen un costo fijo de adquisición y un costo bajo de operación en contraste con los servicios de la nube \citep{FogBonomi,AzurePrice}.

\section{Enriquecimiento de la Información}

Los datos en crudo de una o múltiples fuentes no aporta la suficiente información para derivar conclusiones en un modelo cuyo objetivo es realizar el análisis de los datos. El procesamiento de datos en el borde permite al desarrollador analizar muchos más detalles, tomar decisiones o incluso observar y predecir el comportamiento de las variables implicadas en la medición, con lo cual el desarrollador obtiene una visión más amplia de lo que ocurre en el medio observado \citep{perera2014context,mahmoudi2015object,abrardo2017information}.

%mencionar fusion ann escenario deteccion de inundacion
\subsection{Fusión de Datos}

La fusión de datos como práctica de enriquecimiento de la información es bien conocida. Esta permite al desarrollador ampliar el espectro de la información disponible a partir de diferentes fuentes de información; estas pueden ser diferentes tipos de datos, sensores o incluso imágenes. En el caso de un modelo de detección de inundación es posible usar la información proveniente de diferentes nodos de medición a lo largo de un río y combinarlos usando Redes Neuronales Artificiales para determinar el riesgo de inundación dentro de una ventana de tiempo específica \cite{ANN1,ANN2,ANN3}.

Con esto, una plataforma de procesamiento de borde equipada con software y hardware especializado \iffalse como el propuesto en este proyecto \fi se encuentra en la capacidad de procesar grandes bloques de datos provenientes de campo, siendo capaz de ajustar, según la necesidad de procesamiento los recursos de hardware con el objetivo de reducir la cantidad de datos enviados a la nube, reducir la los volúmenes de almancenamiento, reducir la latencia al actuar directamente sobre el origen de los datos y a su vez facilitar al desarrollador el procesamiento y analítica aplicable a los datos.