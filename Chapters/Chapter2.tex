% Chapter 2

\chapter{Definición del Problema} % Write in your own chapter title
\label{Chapter2}
\lhead{Capítulo 2. \emph{Definición del Problema}} % Write in your own chapter title to set the page header

%tek el de abajo es el caso de estudio; no es el problema como tal...

Determinar de forma rápida y eficaz la solución a un algoritmo complejo visto desde una óptica de IoT implica tener en consideración múltiples factores. Uno de ellos es la potencia de cálculo por unidad de potencia medida en Tera Flops/W y la cantidad de datos a analizar sin que se saturen los recursos disponibles en el diseño de la arquitectura de IoT. Un esquema de operación simple basado en Internet of Things podría no cumplir con los requisitos necesarios para la operación dependiendo del volumen de los datos y el tipo de procesamiento que se realice sobre los mismos; relegar todo este procesamiento a la nube podría ser simplemente muy costoso y/o imposible de implementar debido a limitaciones netamente técnicas.

%Dado el alto riesgo de inundación que presentan algunos de los ríos que surten de agua potable a las ciudades y poblaciones aledañas a su cuenca, existe la necesidad de monitorear y predecir su comportamiento en el tiempo a partir de la información meteorológica y del sensado de las variables físicas disponibles, con el fin de tomar acciones y alertar la población tempranamente.

%Dar solución a un problema como este desde un punto de vista de IoT implica analizar diferentes dimensiones, una de ellas es la capacidad de procesamiento y analítica que lleva tras de sí predecir y tomar decisiones. Por otro lado, tenemos los costos que implica desplegar una solución con la instrumentación y máquinas necesaria para procesar grandes bloques de información en la nube.

\section{Procesamiento de la Información}

Resolver el problema de la potencia de procesamiento desde un esquema de IoT ha traído consigo múltiples retos que no pueden ser resueltos por un esquema de operación basado en \textit{Cloud Computing}, es decir, que problemas como la alta latencia, la pérdida de conexión entre la fuente y el repositorio de datos (pérdida de información), el uso del ancho de banda y la administración de los recursos son inevitables, a menos que la aplicación sea insensible a ellos. Uno de los problemas más comunes es claramente la latencia; esta implica enviar grandes cantidades de información a través de Internet usando un ancho de banda determinado puede retrasar el procesamiento hasta que toda la información esté disponible en el servidor de la nube; este problema se suma al tiempo que tarda en responder el servidor, imponiendo unas restricciones en el tiempo de respuesta mínimo que define en \citep{RTSSystems}.

%tek cuáles problemas son inevitables?

\section{Costo del Procesamiento y Almacenamiento}

El procesamiento centralizado ofrecido por los proveedores de computación en la nube es robusto en general, pero puede verse limitado su uso en el tiempo para cierto tipo de aplicaciones a nivel presupuestal, dado que el despliegue y escalamiento de gran cantidad de servidores y los altos niveles de tráfico de red hacia los servidores implican gastos que deben ser cubiertos según el tipo de contrato que se lleve a cabo con el proveedor que de otra forma requeririan la adquisición de nuevo hardware.

Por otra parte, el almacenamiento en la nube también implica altos costos, originados tanto por la persistencia de los datos como por la cantidad de transacciones que se realizan al medio de almacenamiento \citep{AzurePrice,AWSPrice,GCPrice,UPrice}.

Con todo esto, se hace necesario encontrar soluciones capaces de  analizar, transformar, tomar decisiones y enviar gran cantidad de información conservando el TDP característico de los procesadores orientados a IoT.
