% Chapter 14

\chapter{Conclusiones} % Write in your own chapter title
\label{Chapter14}
\lhead{Capítulo 14. \emph{Conclusiones}} % Write in your own chapter title to set the page header

%primera conclusion sobre el objetivo principal del proyecto: viabilidad de un acelerador de borde

La transformación de altos volúmenes de datos es viable usando dispositivos destinados orientados al Internet de las Cosas con una baja potencia computacional usando hardware. Con un diseño de hardware y las optimizaciones propuestas es posible realizar múltiples operaciones sobre los datos concurrentemente con un bajo footprint sobre el procesador, relegando únicamente la carga operativa a la FPGA y dejando que el procesador atienda los diferentes servicios de red que implican una conexión con la Nube.

Se pudo determinar que es posible implementar directamente el modelo Hardware usando como base un modelo Software con herramientas como Matlab sin necesidad de implementar directamente los complejos algoritmos de entrenamiento sobre el Hardware usando el lenguaje de síntesis de alto nivel, en este caso C/C++.%poner por qué es una alternativa a sysgen

El diseño de hardware propuesto proporciona una ganancia de 4x la velocidad de cálculo sobre su contraparte software con el mismo diseño base, convirtiéndolo en una solución mucho más eficiente para el procesamiento de los datos en el borde de una red IoT.

Con las optimizaciones de área correctas es posible integrar múltiples módulos IP que realicen diferentes operaciones y transformaciones sobre los datos, sacrificando rendimiento que puede o no ser notable, dependiento de la complejidad y optimizaciones de hardware que requiera el circuito.

Durante la etapa de desarrollo inicial del hardware se determinó que el canal de datos de alta velocidad ofrecido por la interfaz AXI Full es el requerido para una aplicación donde el alto volumen de datos a procesar es un factor determinante para el rendimiento total de la solución. Esto se evidencia al implementar la solución con la interfaz AXI Lite, con la cual se obtuvo un rendimiento más bajo que la solución netamente software.

Se estableció que el ancho en bits del bus de datos del circuito hardware no influye sobre la latencia total del circuito, con lo cual se pudo implementar un ancho de datos de 32 bits, garantizando así una alta resolución para las operaciones sin perder rendimiento.

La implementación de una solución como la propuesta requiere de pasos muy precisos en la configuración del kernel, más aún si se desean características especificas como un sistema de archivos de Ubuntu o módulos de hardware como una DMA para video o para mover grandes bloques de datos de una región a otra de memoria. Con una arquitectura como la propuesta es posible configurar la FPGA en tiempo de ejecución del Kernel de Linux y adicionar nuevos binarios y configuraciones al sistema de archivos usando un servidor sftp y un cliente como FileZilla.
