% Chapter 8

\chapter{Metodología} % Write in your own chapter title
\label{Chapter8}
\lhead{Capítulo 8. \emph{Metodología}} % Write in your own chapter title to set the page header

Para el desarrollo de  los objetivos se seguirán los siguientes lineamientos.
\begin{enumerate}
\item \textbf{Aplicación de la computación consciente del contexto}
\begin{itemize}
\item \textit{Aplicación de algoritmo de procesamiento de datos y escalamiento de hardware}: Una vez definidos los algoritmos a usar, se desarrollarán inicialmente en un lenguaje de programación como C/C++, los cuales servirán como insumo para el producto final usando HLS.
\item \textit{Comparación de algoritmos}: Se realizará una comparación de los algoritmos según la cantidad de sensores a fusionar.
\item \textit{Escritura}: Documentar los hallazgos encontrados.
\end{itemize}

\item \textbf{Elección de la plataforma de computación para IoT que permita reconfiguración parcial dinámica}
\begin{itemize}
\item \textit{Búsqueda tarjeta de desarrollo}: Se consultan las diferentes tarjetas de desarrollo hardware disponibles con programación en alto nivel.

\item \textit{Verificación}: Se realiza un análisis de compatibilidad entre lenguajes de programación y herramientas disponibles.
\item \textit{Escritura}: Se especifica la arquitectura y herramientas a usar.
\end{itemize}
\item \textbf{Integración de los diferentes módulos dentro de la plataforma de procesamiento usando síntesis de alto nivel}
\begin{itemize}
\item \textit{Programación HLS}: Se programan los algoritmos de fusión usando síntesis de alto nivel HLS.
\item \textit{Programación RP}: Se realiza la configuración de los módulos que harán uso de la reconfiguración parcial.
\item \textit{Escritura}: Se reportan los recursos y métodos de hardware usados para la implementación.
\end{itemize}
\item \textbf{Comparar el rendimiento de la plataforma desarrollada con una implementación diseñada completamente en software}
\begin{itemize}
\item \textit{Pruebas iniciales}: Se realiza la verificación de funcionamiento del sistema completo.
\item \textit{Correcciones}: Se corrigen posibles errores de implementación y diseño.
\item \textit{Pruebas finales}: Se realizan pruebas de rendimiento y funcionalidad del sistema en su última etapa.
\item \textit{Evaluación final}: Se evalúa el desempeño del sistema en su totalidad, con lo cual se obtienen los resultados a reportar en el documento.
\item \textit{Escritura}: Se reportan los resultados obtenidos.
\end{itemize}
\end{enumerate}

  