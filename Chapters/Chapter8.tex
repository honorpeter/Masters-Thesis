% Chapter 8

\chapter{Metodología} % Write in your own chapter title
\label{Chapter8}
\lhead{Capítulo 8. \emph{Metodología}} % Write in your own chapter title to set the page header

Para el desarrollo de  los objetivos se seguirán los siguientes lineamientos.
\begin{enumerate}
\item \textbf{Definir los requerimientos y los lineamientos requeridos por la plataforma}
\begin{itemize}
\item Se definen los recursos de hardware necesarios para establecer que tarjeta de desarrollo es suficiente para implementar una plataforma de computación de borde.
\item \textit{Escritura}: Documentar los hallazgos encontrados.
\end{itemize}

\item \textbf{Desarrollar un módulo de fusión de datos sintetizable en hardware, instrumentado y empaquetado como módulo IP}

\begin{itemize}
\item Se desarrollará un módulo de fusión de datos de alto rendimiento usando ANN que permita procesar grandes bloques de información proveniente de múltiples fuentes de datos.
\item \textit{Verificación}: Se realiza un análisis de compatibilidad entre lenguajes de programación y herramientas disponibles.
\item \textit{Escritura}: Se detalla el proceso de diseño e implementación sobre el hardware.
\end{itemize}

\item \textbf{Integrar e implementar los módulos IP de fusión de datos como plataforma consciente del contexto usando reconfiguración parcial}
\begin{itemize}
\item \textit{Programación HLS}: Se programan los algoritmos de fusión usando síntesis de alto nivel HLS, añadiendo múltiples optimizaciones de hardware según sea necesario con el objetivo de mejorar la concurrencia y disminuir la latencia total del circuito.
\item \textit{Programación RP}: Se realiza la configuración de los módulos que harán uso de la reconfiguración parcial.
\item \textit{Escritura}: Se reportan los recursos y métodos de hardware usados para la implementación.
\end{itemize}
\item \textbf{Comparar el rendimiento de la plataforma desarrollada con una implementación diseñada completamente en software}
\begin{itemize}
\item \textit{Pruebas iniciales}: Se realiza la verificación de funcionamiento del sistema completo.
\item \textit{Correcciones}: Se corrigen posibles errores de implementación y diseño.
\item \textit{Pruebas finales}: Se realizan pruebas de rendimiento y funcionalidad del sistema en su última etapa.
\item \textit{Evaluación final}: Se evalúa el desempeño del sistema en su totalidad, con lo cual se obtienen los resultados a reportar en el documento.
\item \textit{Escritura}: Se reportan los resultados obtenidos.
\end{itemize}
\end{enumerate}

  