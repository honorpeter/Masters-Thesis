% Chapter 1

\chapter{Introducción} % Write in your own chapter title
\label{Chapter1}
\lhead{Capítulo 1. \emph{Introducción}} % Write in your own chapter title to set the page header

La computación en la nube como medio de procesamiento y análisis se convirtió en una herramienta con recursos casi ilimitados para los desarrolladores desde su masificación hace unos pocos años. La computación en la nube cuenta con múltiples bondades; entre ellas encontramos el Internet de las Cosas (\textit{IoT}) como una de las aplicaciones más importantes dada la gran cantidad de datos y recursos necesarios para procesar los altos volúmenes de información generados por las redes de sensores a los que usualmente se encuentra acoplada. Hoy, encontramos el término computación de borde, que aunque es un esquema relativamente nuevo de computación, ha llamado la atención de múltiples investigadores dado que implica llevar parte de los servicios de la nube al lugar donde realmente se producen los datos \cite{bonomi2014fog}. Realizar parte del procesamiento en el borde le permite al desarrollador acceder a aplicaciones sensibles a la latencia incapaces de operar directamente sobre la nube, reducir costos en analítica, volumen de datos procesados en la nube, almacenamiento, entre otros. 

%tek

Aparte de las diferentes virtudes que ofrece la computación de borde, también encontramos marcadas limitaciones como la reducida potencia de cálculo de algunos de los dispositivos. En este sentido, la aceleración por hardware y la reconfiguración parcial dinámica, juegan un rol importante, aportando potencia de cálculo en un espacio reducido en función de las necesidades de procesamiento requeridas por la aplicación.

