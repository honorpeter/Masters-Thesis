\chapter{Elección de la Plataforma de Desarrollo} % Write in your own chapter title
\label{Chapter8.5}
\lhead{Capítulo 9. \emph{Plataforma}} % Write in your own chapter title to set the page header

\section{Elección de la plataforma}

Se realiza un análisis primario con las familias de FPGA disponibles comercialmente por Xilinx que permiten reconfiguración dinámica y la instrumentación necesaria para servir como Gateway de IoT como la posibilidad de correr un SO y conexión a red.

La Tabla \ref{fpgaFamilies} muestra las familias de SoC dentro de la categoría de precio reducido y rendimiento medio. Se nota que por cantidad de recursos la opción mas viable es la familia ZYNQ 7000, ofreciendo hasta el doble de celdas lógicas y casi tres veces la cantidad de DSPs que otras plataformas dentro de su categoría, volviéndola rápidamente la familia a examinar.

\begin{table}[h!]
\centering
\resizebox{\textwidth}{!}{%
\begin{tabular}{|l|c|c|c|}
\hline
\multicolumn{1}{|c|}{\textbf{SoC}} & \textbf{ZYNQ 7K} & \textbf{ARTIX 7} & \textbf{SPARTAN 7} \\ \hline
MAX LOGIC CELLS [k] & 444 & 215 & 102 \\ \hline
MAX MEM [Mb] & 26.5 & 13 & 4.2 \\ \hline
DSPs & 2020 & 740 & 160 \\ \hline
TRANSCEIVER SPEEDS [Gb/s] & 12.5 & 6.6 & NA \\ \hline
MAX I/O PINS & 250 & 500 & 400 \\ \hline
\end{tabular}%
}
\caption{Familias de FPGA}
\label{fpgaFamilies}
\end{table}

La Tabla \ref{fpgaComparison} muestra las diferentes plataformas de desarrollo  disponibles y de fabricantes como AVNET y DIGILENT. En ella vemos que los diferentes SoC de cada una de las plataformas cuentan con diferentes recursos de hardware y características extra adicionadas por el fabricante.

\begin{table}[h!]
\small
\centering
\begin{adjustbox}{width=1.1\textwidth}
\small
\begin{tabular}{|l|c|c|c|c|c|c|c|c|c|c|}
\hline
\multicolumn{1}{|c|}{\textbf{Nombre}} & \textbf{X Z7K EKIT} & \textbf{MICROZED} & \textbf{PICOZED} & \textbf{ZEDBOARD} & \textbf{MINIZED} & \textbf{PYNQ-Z1} & \textbf{ZYBO N} & \textbf{ZYBO O} \\ \hline
\textbf{SoC} & ZC702 & ZC7010 & ZCZ010 & XC7Z020 & XC7Z007S & XC7Z020 & XC7Z020 & XC7Z010 \\ \hline
\textbf{Price} & 895 & 266 & 178 & 475 & 89 & 199 & 299 & 199 \\ \hline
\textbf{Max LogicCells} & 85000 & 28000 & 28000 & 85000 & 23000 & 53200 & 53200 & 28000 \\ \hline
\textbf{Max BRAM Mb} & 4.9 & 1.92 & 1.92 & 4.9 & 1.8 & 4.9 & 4.9 & 1.92 \\ \hline
\textbf{DSPs} & 220 & 80 & 80 & 220 & 66 & 220 & 220 & 80 \\ \hline
\textbf{DDR RAM MB} & 1024 & 1024 & 1024 & 512 & 512 & 512 & 1024 & 512 \\ \hline
\textbf{Notas} &  &  & No headers &  & No headers &  &  &  \\ \hline
\end{tabular}%
\end{adjustbox}
\caption{Comparación tarjetas de desarrollo}
\label{fpgaComparison}
\end{table}

Para elegir la plataforma de desarrollo ideal se debe tener en cuenta el objetivo de lo que se está desarrollando y más importante aún, el para qué se está desarrollando. En este caso, se desea implementar un algoritmo altamente concurrente con el objetivo de operar bajo un esquema de IoT. Por lo tanto, es importante tener en cuenta que se requiere que los recursos de hardware sean altos, que tenga un stack de red y los jacks de conexión, todo esto bajo un kernel de Linux que permita administración remota usando SSH. Desde esta óptica, la mayoría de tarjetas cumplen con los requisitos. Algunas, como las Picozed y Minized, del fabricante AVNet no cuentan con el jack RJ45 de red, entre otros headers; para estas tarjetas, es necesario adicionar accesorios que incrementarían el precio base. Finalmente, se debe tener en cuenta el precio y el soporte del fabricante. En este caso, basado en la cantidad de recursos y el precio, la mejor plataforma sería la \textbf{PYNQ-Z1}. Aunque cuenta con 512MB de RAM DDR3, por su precio es imbatible frente a las otras plataformas; de requerirse más memoria RAM, la mejor opción sería la versión nueva de ZYBO, ambas del fabricante DIGILENT.

Dentro de este esquema de desarrollo es necesario aclarar que independientemente de la plataforma de desarrollo elegida, la implementación debería ser transversal a todas las familias de Xilinx que usen VIVADO como sintetizador de hardware, siendo las diferencias entre los SoC la velocidad de transferencia y eficiencia de la FPGA en el diseño de sus bloques lógicos, pudiendo obtener latencias más bajas en FPGAs más costosas.
